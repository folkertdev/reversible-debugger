% This is samplepaper.tex, a sample chapter demonstrating the
% LLNCS macro package for Springer Computer Science proceedings;
% Version 2.20 of 2017/10/04
%
\documentclass[runningheads]{llncs}
%
\usepackage{graphicx,hyperref}
% Used for displaying a sample figure. If possible, figure files should
% be included in EPS format.
%
% If you use the hyperref package, please uncomment the following line
% to display URLs in blue roman font according to Springer's eBook style:
% \renewcommand\UrlFont{\color{blue}\rmfamily}

\begin{document}
%
\title{Reversible Session-Based Concurrency \\ in Haskell\thanks{F.\ de Vries is a BSc student.}}
%
%\titlerunning{Abbreviated paper title}
% If the paper title is too long for the running head, you can set
% an abbreviated paper title here
%
\author{Folkert de Vries\inst{1} \and
Jorge A. P\'{e}rez\inst{1}\orcidID{0000-0002-1452-6180}}
%
\authorrunning{F.\ de Vries and J.\ A.\ P\'{e}rez}
% First names are abbreviated in the running head.
% If there are more than two authors, 'et al.' is used.
%
\institute{University of Groningen, The Netherlands}
%
\maketitle              % typeset the header of the contribution
%
\begin{abstract}
Under a reversible semantics, computation steps can be undone. 
For message-passing, concurrent programs, reversing computation steps is a challenging and delicate task; one typically aims at formal semantics which are \emph{causally-consistent}. 
Prior work has addressed this challenge in the context of a process model of multiparty protocols (choreographies) following a so-called \emph{monitors-as-memories} approach.
In this paper, we describe our ongoing efforts aimed at implementing this operational semantics in Haskell. 
\keywords{Reversible computation \and Message-passing concurrency \and Session Types \and Haskell.}
\end{abstract}
%
%
%
\section{Introduction}
We implement the model in~\cite{DBLP:conf/ppdp/MezzinaP17}.

\section{The Process Model}

\section{Our Haskell Implementation}

\section{Concluding Remarks and Future Work}
%
% ---- Bibliography ----
%
% BibTeX users should specify bibliography style 'splncs04'.
% References will then be sorted and formatted in the correct style.
%
 \bibliographystyle{splncs04}
 \bibliography{paper}
%
\end{document}
